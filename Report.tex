%% bare_conf.tex
%% V1.3
%% 2007/01/11
%% by Michael Shell
%% See:
%% http://www.michaelshell.org/
%% for current contact information.
%%
%% This is a skeleton file demonstrating the use of IEEEtran.cls
%% (requires IEEEtran.cls version 1.7 or later) with an IEEE conference paper.
%%\\
%% Support sites:
%% http://www.michaelshell.org/tex/ieeetran/
%% http://www.ctan.org/tex-archive/macros/latex/contrib/IEEEtran/
%% and
%% http://www.ieee.org/

%%*************************************************************************
%% Legal Notice:6
%% This code is offered as-is without any warranty either expressed or
%% implied; without even the implied warranty of MERCHANTABILITY or
%% FITNESS FOR A PARTICULAR PURPOSE! 
%% User assumes all risk.
%% In no event shall IEEE or any contributor to this code be liable for
%% any damages or losses, including, but not limited to, incidental,
%% consequential, or any other damages, resulting from the use or misuse
%% of any information contained here.
%%
%% All comments are the opinions of their respective authors and are not
%% necessarily endorsed by the IEEE.
%%
%% This work is distributed under the LaTeX Project Public License (LPPL)
%% ( http://www.latex-project.org/ ) version 1.3, and may be freely used,
%% distributed and modified. A copy of the LPPL, version 1.3, is included
%% in the base LaTeX documentation of all distributions of LaTeX released
%% 2003/12/01 or later.
%% Retain all contribution notices and credits.
%% ** Modified files should be clearly indicated as such, including  **
%% ** renaming them and changing author support contact information. **
%%
%% File list of work: IEEEtran.cls, IEEEtran_HOWTO.pdf, bare_adv.tex,
%%                    bare_conf.tex, bare_jrnl.tex, bare_jrnl_compsoc.tex
%%*************************************************************************

% *** Authors should verify (and, if needed, correct) their LaTeX system  ***
% *** with the testflow diagnostic prior to trusting their LaTeX platform ***
% *** with production work. IEEE's font choices can trigger bugs that do  ***
% *** not appear when using other class files.                            ***
% The testflow support page is at:
% http://www.michaelshell.org/tex/testflow/



% Note that the a4paper option is mainly intended so that authors in
% countries using A4 can easily print to A4 and see how their papers will
% look in print - the typesetting of the document will not typically be
% affected with changes in paper size (but the bottom and side margins will).
% Use the testflow package mentioned above to verify correct handling of
% both paper sizes by the user's LaTeX system.
%
% Also note that the "draftcls" or "draftclsnofoot", not "draft", option
% should be used if it is desired that the figures are to be displayed in
% draft mode.
%
\documentclass[conference]{IEEEtran}
\usepackage{mathrsfs}
\usepackage{amsfonts}
\usepackage{mathtools}
\usepackage{amsbsy}
\usepackage[ruled,vlined]{algorithm2e}
\usepackage{accents}
\makeatletter
\newcommand{\ubar}[1]{\underaccent{\bar}{#1}}
\newcommand{\removelatexerror}{\let\@latex@error\@gobble}
\newcommand{\nosemic}{\renewcommand{\@endalgocfline}{\relax}}% Drop semi-colon ;
\newcommand{\dosemic}{\renewcommand{\@endalgocfline}{\algocf@endline}}% Reinstate semi-colon ;
\newcommand{\pushline}{\Indp}% Indent
\newcommand{\popline}{\Indm\dosemic}% Undent
\let\oldnl\nl% Store \nl in \oldnl
\newcommand{\nonl}{\renewcommand{\nl}{\let\nl\oldnl}}% Remove line number for one line	
\makeatother% Add the compsoc option for Computer Society conferences.


%
% If IEEEtran.cls has not been installed into the LaTeX system files,
% manually specify the path to it like:
% \documentclass[conference]{../sty/IEEEtran}





% Some very useful LaTeX packages include:
% (uncomment the ones you want to load)


% *** MISC UTILITY PACKAGES ***
%
%\usepackage{ifpdf}
% Heiko Oberdiek's ifpdf.sty is very useful if you need conditional
% compilation based on whether the output is pdf or dvi.
% usage:
% \ifpdf
%   % pdf code
% \else
%   % dvi code
% \fi
% The latest version of ifpdf.sty can be obtained from:
% http://www.ctan.org/tex-archive/macros/latex/contrib/oberdiek/
% Also, note that IEEEtran.cls V1.7 and later provides a builtin
% \ifCLASSINFOpdf conditional that works the same way.
% When switching from latex to pdflatex and vice-versa, the compiler may
% have to be run twice to clear warning/error messages.






% *** CITATION PACKAGES ***
%
%\usepackage{cite}
% cite.sty was written by Donald Arseneau
% V1.6 and later of IEEEtran pre-defines the format of the cite.sty package
% \cite{} output to follow that of IEEE. Loading the cite package will
% result in citation numbers being automatically sorted and properly
% "compressed/ranged". e.g., [1], [9], [2], [7], [5], [6] without using
% cite.sty will become [1], [2], [5]--[7], [9] using cite.sty. cite.sty's
% \cite will automatically add leading space, if needed. Use cite.sty's
% noadjust option (cite.sty V3.8 and later) if you want to turn this off.
% cite.sty is already installed on most LaTeX systems. Be sure and use
% version 4.0 (2003-05-27) and later if using hyperref.sty. cite.sty does
% not currently provide for hyperlinked citations.
% The latest version can be obtained at:
% http://www.ctan.org/tex-archive/macros/latex/contrib/cite/
% The documentation is contained in the cite.sty file itself.






% *** GRAPHICS RELATED PACKAGES ***
%
\ifCLASSINFOpdf
  % \usepackage[pdftex]{graphicx}
  % declare the path(s) where your graphic files are
  % \graphicspath{{../pdf/}{../jpeg/}}
  % and their extensions so you won't have to specify these with
  % every instance of \includegraphics
  % \DeclareGraphicsExtensions{.pdf,.jpeg,.png}
\else
  % or other class option (dvipsone, dvipdf, if not using dvips). graphicx
  % will default to the driver specified in the system graphics.cfg if no
  % driver is specified.
  % \usepackage[dvips]{graphicx}
  % declare the path(s) where your graphic files are
  % \graphicspath{{../eps/}}
  % and their extensions so you won't have to specify these with
  % every instance of \includegraphics
  % \DeclareGraphicsExtensions{.eps}
\fi
% graphicx was written by David Carlisle and Sebastian Rahtz. It is
% required if you want graphics, photos, etc. graphicx.sty is already
% installed on most LaTeX systems. The latest version and documentation can
% be obtained at: 
% http://www.ctan.org/tex-archive/macros/latex/required/graphics/
% Another good source of documentation is "Using Imported Graphics in
% LaTeX2e" by Keith Reckdahl which can be found as epslatex.ps or
% epslatex.pdf at: http://www.ctan.org/tex-archive/info/
%
% latex, and pdflatex in dvi mode, support graphics in encapsulated
% postscript (.eps) format. pdflatex in pdf mode supports graphics
% in .pdf, .jpeg, .png and .mps (metapost) formats. Users should ensure
% that all non-photo figures use a vector format (.eps, .pdf, .mps) and
% not a bitmapped formats (.jpeg, .png). IEEE frowns on bitmapped formats
% which can result in "jaggedy"/blurry rendering of lines and letters as
% well as large increases in file sizes.
%
% You can find documentation about the pdfTeX application at:
% http://www.tug.org/applications/pdftex





% *** MATH PACKAGES ***
%
%\usepackage[cmex10]{amsmath}
% A popular package from the American Mathematical Society that provides
% many useful and powerful commands for dealing with mathematics. If using
% it, be sure to load this package with the cmex10 option to ensure that
% only type 1 fonts will utilized at all point sizes. Without this option,
% it is possible that some math symbols, particularly those within
% footnotes, will be rendered in bitmap form which will result in a
% document that can not be IEEE Xplore compliant!
%
% Also, note that the amsmath package sets \interdisplaylinepenalty to 10000
% thus preventing page breaks from occurring within multiline equations. Use:
%\interdisplaylinepenalty=2500
% after loading amsmath to restore such page breaks as IEEEtran.cls normally
% does. amsmath.sty is already installed on most LaTeX systems. The latest
% version and documentation can be obtained at:
% http://www.ctan.org/tex-archive/macros/latex/required/amslatex/math/





% *** SPECIALIZED LIST PACKAGES ***
%
%\usepackage{algorithmic}
% algorithmic.sty was written by Peter Williams and Rogerio Brito. This
% package provides an algorithmic environment fo describing algorithms. You
% can use the algorithmic environment in-text or within a figure environment
% to provide for a floating algorithm. Do NOT use the algorithm floating
% environment provided by algorithm.sty (by the same authors) or
% algorithm2e.sty (by Christophe Fiorio) as IEEE does not use dedicated
% algorithm float types and packages that provide these will not provide
% correct IEEE style captions. The latest version and documentation of
% algorithmic.sty can be obtained at: http://www.ctan.org/tex-
% archive/macros/latex/contrib/algorithms/ There is also a support site at:
% http://algorithms.berlios.de/index.html Also of interest may be the
% (relatively newer and more customizable) algorithmicx.sty package by Szasz
% Janos: http://www.ctan.org/tex-archive/macros/latex/contrib/algorithmicx/




% *** ALIGNMENT PACKAGES ***
%
%\usepackage{array}
% Frank Mittelbach's and David Carlisle's array.sty patches and improves
% the standard LaTeX2e array and tabular environments to provide better
% appearance and additional user controls. As the default LaTeX2e table
% generation code is lacking to the point of almost being broken with
% respect to the quality of the end results, all users are strongly
% advised to use an enhanced (at the very least that provided by array.sty)
% set of table tools. array.sty is already installed on most systems. The
% latest version and documentation can be obtained at:
% http://www.ctan.org/tex-archive/macros/latex/required/tools/

%\usepackage{mdwmath}
%\usepackage{mdwtab}
% Also highly recommended is Mark Wooding's extremely powerful MDW tools,
% especially mdwmath.sty and mdwtab.sty which are used to format equations
% and tables, respectively. The MDWtools set is already installed on most
% LaTeX systems. The lastest version and documentation is available at:
% http://www.ctan.org/tex-archive/macros/latex/contrib/mdwtools/


% IEEEtran contains the IEEEeqnarray family of commands that can be used to
% generate multiline equations as well as matrices, tables, etc., of high
% quality.


%\usepackage{eqparbox}
% Also of notable interest is Scott Pakin's eqparbox package for creating
% (automatically sized) equal width boxes - aka "natural width parboxes".
% Available at:
% http://www.ctan.org/tex-archive/macros/latex/contrib/eqparbox/





% *** SUBFIGURE PACKAGES ***
%\usepackage[tight,footnotesize]{subfigure}
% subfigure.sty was written by Steven Douglas Cochran. This package makes it
% easy to put subfigures in your figures. e.g., "Figure 1a and 1b". For IEEE
% work, it is a good idea to load it with the tight package option to reduce
% the amount of white space around the subfigures. subfigure.sty is already
% installed on most LaTeX systems. The latest version and documentation can
% be obtained at:
% http://www.ctan.org/tex-archive/obsolete/macros/latex/contrib/subfigure/
% subfigure.sty has been superceeded by subfig.sty.



%\usepackage[caption=false]{caption}
%\usepackage[font=footnotesize]{subfig}
% subfig.sty, also written by Steven Douglas Cochran, is the modern
% replacement for subfigure.sty. However, subfig.sty requires and
% automatically loads Axel Sommerfeldt's caption.sty which will override
% IEEEtran.cls handling of captions and this will result in nonIEEE style
% figure/table captions. To prevent this problem, be sure and preload
% caption.sty with its "caption=false" package option. This is will preserve
% IEEEtran.cls handing of captions. Version 1.3 (2005/06/28) and later 
% (recommended due to many improvements over 1.2) of subfig.sty supports
% the caption=false option directly:
%\usepackage[caption=false,font=footnotesize]{subfig}
%
% The latest version and documentation can be obtained at:
% http://www.ctan.org/tex-archive/macros/latex/contrib/subfig/
% The latest version and documentation of caption.sty can be obtained at:
% http://www.ctan.org/tex-archive/macros/latex/contrib/caption/




% *** FLOAT PACKAGES ***
%
%\usepackage{fixltx2e}
% fixltx2e, the successor to the earlier fix2col.sty, was written by
% Frank Mittelbach and David Carlisle. This package corrects a few problems
% in the LaTeX2e kernel, the most notable of which is that in current
% LaTeX2e releases, the ordering of single and double column floats is not
% guaranteed to be preserved. Thus, an unpatched LaTeX2e can allow a
% single column figure to be placed prior to an earlier double column
% figure. The latest version and documentation can be found at:
% http://www.ctan.org/tex-archive/macros/latex/base/



%\usepackage{stfloats}
% stfloats.sty was written by Sigitas Tolusis. This package gives LaTeX2e
% the ability to do double column floats at the bottom of the page as well
% as the top. (e.g., "\begin{figure*}[!b]" is not normally possible in
% LaTeX2e). It also provides a command:
%\fnbelowfloat
% to enable the placement of footnotes below bottom floats (the standard
% LaTeX2e kernel puts them above bottom floats). This is an invasive package
% which rewrites many portions of the LaTeX2e float routines. It may not work
% with other packages that modify the LaTeX2e float routines. The latest
% version and documentation can be obtained at:
% http://www.ctan.org/tex-archive/macros/latex/contrib/sttools/
% Documentation is contained in the stfloats.sty comments as well as in the
% presfull.pdf file. Do not use the stfloats baselinefloat ability as IEEE
% does not allow \baselineskip to stretch. Authors submitting work to the
% IEEE should note that IEEE rarely uses double column equations and
% that authors should try to avoid such use. Do not be tempted to use the
% cuted.sty or midfloat.sty packages (also by Sigitas Tolusis) as IEEE does
% not format its papers in such ways.





% *** PDF, URL AND HYPERLINK PACKAGES ***
%
%\usepackage{url}
% url.sty was written by Donald Arseneau. It provides better support for
% handling and breaking URLs. url.sty is already installed on most LaTeX
% systems. The latest version can be obtained at:
% http://www.ctan.org/tex-archive/macros/latex/contrib/misc/
% Read the url.sty source comments for usage information. Basically,
% \url{my_url_here}.





% *** Do not adjust lengths that control margins, column widths, etc. ***
% *** Do not use packages that alter fonts (such as pslatex).         ***
% There should be no need to do such things with IEEEtran.cls V1.6 and later.
% (Unless specifically asked to do so by the journal or conference you plan
% to submit to, of course. )


% correct bad hyphenation here
\hyphenation{op-tical net-works semi-conduc-tor}


\begin{document}
%
% paper title
% can use linebreaks \\ within to get better formatting as desired
\title{Approximate Inference}


% author names and affiliations
% use a multiple column layout for up to three different
% affiliations
\author{\IEEEauthorblockN{Zhiwei Han}
\IEEEauthorblockA{Faculty of Electrical Engineering and Information Technology\\
Technical University of Munich\\
Arcisstr. 21, Munich, 80333\\
Email: hanzw356255531@icloud.com}
}

% conference papers do not typically use \thanks and this command
% is locked out in conference mode. If really needed, such as for
% the acknowledgment of grants, issue a \IEEEoverridecommandlockouts
% after \documentclass

% for over three affiliations, or if they all won't fit within the width
% of the page, use this alternative format:
% 
%\author{\IEEEauthorblockN{Michael Shell\IEEEauthorrefmark{1},
%Homer Simpson\IEEEauthorrefmark{2},
%James Kirk\IEEEauthorrefmark{3}, 
%Montgomery Scott\IEEEauthorrefmark{3} and
%Eldon Tyrell\IEEEauthorrefmark{4}}
%\IEEEauthorblockA{\IEEEauthorrefmark{1}School of Electrical and Computer Engineering\\
%Georgia Institute of Technology,
%Atlanta, Georgia 30332--0250\\ Email: see http://www.michaelshell.org/contact.html}
%\IEEEauthorblockA{\IEEEauthorrefmark{2}Twentieth Century Fox, Springfield, USA\\
%Email: homer@thesimpsons.com}
%\IEEEauthorblockA{\IEEEauthorrefmark{3}Starfleet Academy, San Francisco, California 96678-2391\\
%Telephone: (800) 555--1212, Fax: (888) 555--1212}
%\IEEEauthorblockA{\IEEEauthorrefmark{4}Tyrell Inc., 123 Replicant Street, Los Angeles, California 90210--4321}}




% use for special paper notices
%\IEEEspecialpapernotice{(Invited Paper)}




% make the title area
\maketitle

% The goal of reinforcement learning is to obtain an efficient and universal learning algorithm, which is possible to learn from experimental knowledge and able to react by given enviromental information.
% Temporal-Difference is a traditional and effi
% GQ($\lambda$) is a variation of classic Q-Learning method which applied the gradient decent method to iteratively update the Q-Functions directly(tabularwise algorithm) to minimize the Mean-Squre Projected Bellman Error. This 
% In GQ($\lambda$) and RO-GQ($\lambda$), 
% Besides the Framework, the significant difference between the comparised algorithm is the 
% Even with the lineraly growing of state or action number, the computational expense increases exponentially. Hence, a sparse presentation of Q-Functions would be one smart solution to this kind of problem. 
\begin{abstract}
%\boldmath
This is a seminar work from Chair of Medientechnik, Technical University of Munich. Most of this work is based an the deep learning book from MIT. In this seminar work, an overview of approximation inference and how this technique is applied to solve intractable problem of posterior in probabilistic model, is given. We begin with the introduction of the intractable problems raised in the inference of a probabilistic model and then present a new objective function for the optimization problem. After that, we show the great simplification of the optimization problem with the new objective function instead of the old one with mathematical proof. At the end, a CNN-based variational auto encoder is presented to show that with this technique a deep generative model, whose latent variables are gaussian distributed, can be train in a feasible time.\\
\end{abstract}
% IEEEtran.cls defaults to using nonbold math in the Abstract.
% This preserves the distinction between vectors and scalars. However,
% if the journal you are submitting to favors bold math in the abstract,
% then you can use LaTeX's standard command \boldmath at the very start
% of the abstract to achieve this. Many IEEE journals frown on math
% in the abstract anyway.

% Note that keywords are not normally used for peerreview papers.
\begin{IEEEkeywords}
Approximate Inferience, Variational Inference, Deep Learning, Generative Model, Auto Encoder.
\end{IEEEkeywords}






% For peer review papers, you can put extra information on the cover
% page as needed:
% \ifCLASSOPTIONpeerreview
% \begin{center} \bfseries EDICS Category: 3-BBND \end{center}
% \fi
%
% For peerreview papers, this IEEEtran command inserts a page break and
% creates the second title. It will be ignored for other modes.
\IEEEpeerreviewmaketitle



\section{Introduction}
In many statistical learning problems especially the training of a generative model, inference is considered as the very first step to train the probabilistic models before performing specific optimization method like maximum likelihood learning.  For some simple graphical models like Restricted Boltzmann Machines (RBM) and probabilistic PCA, inference can be simply done by computing the posterior and taking the expectation over it \cite{Goodfellow-et-al-2016}. Those computations are critical process and are also basis of training step afterwards. However, with some graphical models have multiple layers like Deep Belief Network or intractable connections between the latent variables, the exact direct computation of inference in a constraint time will cost  an exponential amount of time. Consequently, a precise evaluation of inference is infeasible because of the explosion of computational complexity and the limited computational power.\\

In the context of deep learning, the problem setting can be organized in a more specific way. We assume that we have a set of visible variables $\boldsymbol{v}$, which can be seen as the input of a one layer RBM and a set of latent variables $\boldsymbol{h}$, the corresponding output. The goal of inference on such model is to compute the posterior $p(\boldsymbol{h} | \boldsymbol{v})$ analytically. Unfortunately, the main challenge is usually the result of the intractable inference problems due to several interactions between latent variables in a structured graphical model. In other words, it's definitely inefficient, if we still calculate the posterior in the traditional way when the latent variables are not independent anymore.\\

One possibility how we deal with these kind of intractable inference problems is variational inference. Instead of trivially integral over the latent variables, we are going to find a distribution to approximate the posterior as much as possible and a lower bound of log likelihood function with respect to this approximate posterior. Finally, maximize this lower bound over model variables. For a perfect approximation $q$ of posterior $p(\boldsymbol{h} | \boldsymbol{v})$, the lower bound is exactly the log likelihood function.\\

The goal of this seminar work is to present an overview about the approximate inference and effective method to confront these issues in term of statistics. In the second section, we show the basic concept of inference and what is the problem need to handle with through an intuitive example. In the third, fourth and fifth section, we introduce several techniques for solving intractable inference problem and learning with structured probabilistic models, whose latent variables are either discrete or continuous. As the learned approximate posterior inference model can be used in a huge amount of tasks, in the last section, we show that after a neural network is used for recognition model, it's turned out to be a $variational$ $auto$-$encoder$.

\section{Background}
\subsection{Inference}
% What is inference and why we need inference
In machine learning community, discriminant method and generative method are two main approaches to solve specific learning tasks with large data sets, their models are therefore named as discriminant models (SVM, Logistic Regression) and generative models (GMM, HMM), respectively. \\

The goal of discriminant models is prediction, in other words the discriminant model learns the \textbf{conditional probability distribution} $p(\boldsymbol{c} | \boldsymbol{o})$, which is the conditional probability of class vector $\boldsymbol{c}$ given observation vector $\boldsymbol{o}$, and the model should be able to predict the exact class of a new coming observation according to a predefined criteria (e.g. the conditional probability is higher than a threshold) afterwards. While the generative model does inference, that is to learn the \textbf{joint distribution} $p(\boldsymbol{c},\boldsymbol{o})$ of the given data sets. Since the generative model knows the joint distribution of the data sets, so conditional probability can be easily derived by dividing the joint distribution with prior according to Bayes rule, 
\begin{equation}
p(\boldsymbol{c} | \boldsymbol{o}) = \frac{p(\boldsymbol{c}, \boldsymbol{o})}{p(\boldsymbol{o})}
\end{equation} From the above example, we can see that inference is a generalization form of prediction. Therefore, generative model has better representation ability and a faster convergence. The drawbacks is that the training of generative model is more computational complex.\\

We define here the problem setting for the rest of this seminar work. Our inference problems are built so that, the models are consisting of visible variables $\boldsymbol{v}$ and latent variables $\boldsymbol{h}$. We would like to maximize likelihood of the given dataset $\boldsymbol{x}$.\\

% Why we could not apply normal inference in intractable models
Since for discriminant model there are already lots of efficient computational algorithms and this seminar work is mainly about approximate inference, in the rest of this seminar work we mainly focus on the application of approximate inference in generative models. Consider the standard training procedure of a generative model, which has visible variables $\boldsymbol{v}$ and latent variables $\boldsymbol{h}$, as first step we need to compute the likelihood by marninalize the its visible variables over latent variable as follows.
\begin{equation}
  L(\boldsymbol{v}\ |\ \boldsymbol{\theta}) = \int_{\boldsymbol{h}}p(\boldsymbol{v} \ |\ \boldsymbol{h}, \boldsymbol{\theta})p(\boldsymbol{h}\ |\ \boldsymbol{\theta})\mathrm{d}\boldsymbol{h}.
\end{equation}
Simple graphical models remain the computation of posterior $p(\boldsymbol{h}\ |\ \boldsymbol{\theta})$ still solvable e.g. RBM (see fig. 1). Unfortunately, most applicable graphical model usually have interactions between their latent variables and thus also have intractable posterior distribution (see fig. 2), it means that the v-structure and the intractable edge between the latent variables$p(\boldsymbol{h}\ |\ \boldsymbol{\theta})$ make the posterior distribution intractable. Consequently, the computational expense rise dramatically and it is almost impossible to finish the computations of posterior in a feasible training time. However, with approximation inference, we are given a powerful weapon and then able to solve this problem.
\begin{figure}[ht]
	\centering
	\includegraphics[scale=0.8]{picture1.png}
     \caption{RBM: Every latent variable is independent to each other since there is no connection between them. The posteriors are through factorization solvable.}\label{pic_1}
\end{figure}

\begin{figure}[ht]
	\centering
	\includegraphics[scale=0.8]{picture2.png}
     \caption{Models with v-structure and edge between latent variables: The posterior distribution are intractable because of the interaction between latent variables}
     \label{pic_2}
\end{figure}
\subsection{MAP Inference}
As shown in (2), when training a probabilistic model e.g. a generative model, we are always interested in computing the data distribution by integral over a set latent variables, namely inference. However, integral over all latent variables could be computationally expensive and should be therefore avoided in the real implementation. A solution to this problem is to computer the most likely latent variable $\boldsymbol{h^*}$, rather than integral over all possible latent variables. Because in practice, for most $\boldsymbol{h}$, $P(\boldsymbol{v} |\boldsymbol{h})$ will be nearly zero, and hence contribute almost nothing to the calculation of likelihood $p(\boldsymbol{v})$.\cite{Goodfellow-et-al-2016}\cite{doersch2016tutorial}\\

Mathmatically, it is equal to an optimization problem as follows,
\begin{equation}
\boldsymbol{h^*}=\underset{\boldsymbol{h}}{\arg\max}p(\boldsymbol{h}\ |\ \theta),
\end{equation}
and approximate (2) as 
\begin{equation}
  \begin{split}
    L(\boldsymbol{v}\ |\ \boldsymbol{\theta}) &=\int_{\boldsymbol{h}}p(\boldsymbol{v} \ |\ \boldsymbol{h}, \boldsymbol{\theta})p(\boldsymbol{h}\ |\ \boldsymbol{\theta})\mathrm{d}\boldsymbol{h}\\
    &=p(\boldsymbol{v}\ |\ \boldsymbol{h^*}, \theta)p(\boldsymbol{h^*}\ |\ \theta)
  \end{split}  
\end{equation}
This method is known as maximum a posteriori inference (MAP).\\


\subsection{EM Algorithm}
Expectation Maximization (EM) \cite{moon1996expectation} algorithm is a standard iterative learning algorithm, which is based on maximum likelihood estimation and especially designed for models with latent variables. \\

EM algorithm includes two steps and runs until the predefined convergent criteria is satisfied,
\subsubsection{Initialization}
Initialize the model parameters $\boldsymbol{\theta}_0$
\subsubsection{Expectation Step}
Compute the objective function (sum of ELBO on all data index) according to (3), 
\begin{equation}
\sum\mathcal{L}(\boldsymbol{v}^{(i)}, \boldsymbol{\theta}, q).
\end{equation}
Note, set $q(\boldsymbol{h}^{(i)}\ |\ \boldsymbol{v})$ for all the index of the data set we need to train on and remain distribution $q(\boldsymbol{h}\ |\ \boldsymbol{v})$ always equal to $p(\boldsymbol{h}\ |\ \boldsymbol{v}, \boldsymbol{\theta}_0)$ while updating $p(\boldsymbol{h}\ |\ \boldsymbol{v}, \boldsymbol{\theta}_t)$ with $\theta_t$, where $t$ is the number of current iteration.
\subsubsection{Maximization Step}
Maximize the objective function (sum of ELBO on all data index) over model parameters $\boldsymbol{\theta}_t$ with arbitrary optimization algorithm.
\subsubsection{Repeat 2), 3) until converge}
\section{Variational Inference}
\subsection{Objective function}
Many difficult sample based inference problems which make use of observations can be reconstructed as optimization problems and they maximize the log-likelihood function of the given datasets.\cite{doersch2016tutorial}\cite{anzai2012pattern} Approximate Inference algorithm will then simplify the underlying optimization problems by using the approximation of posteriors. \\

While the intractability between latent variables make the likelihood computation much more difficult (because of integral), instead of directly calculating and optimizing the log-likelihood, we introduce a new objective function here, which it is easy to compute and optimize if a distribution $q(\boldsymbol{h}\ |\ \boldsymbol{v})$ could be found. This means that we need to find a new distribution $q(\boldsymbol{h}\ |\ \boldsymbol{v})$ which can make a good approximation of the posterior. Or in other words, it is a function, which takes a vector $\boldsymbol{v}$ as visible variable and give us a distribution over the latent variable $\boldsymbol{h}$ that are likely to produce the visible variable $\boldsymbol{v}$. Ideally, the distribution of $\boldsymbol{h}$ under $q(\boldsymbol{h}\ |\ \boldsymbol{v})$ is much simpler than the one under the posterior $p(\boldsymbol{h}\ |\ \boldsymbol{v})$. This trick makes the computation of $\mathbb{E}_{\boldsymbol{h}\sim q}\bigg[p(\boldsymbol{v}\ |\ \boldsymbol{h})\bigg]$ significant much easier.\\

We begin with the derivation of the new objective function from the definition of KL Divergence between the distribution $q(\boldsymbol{h}\ |\ \boldsymbol{v})$ and the posterior $p(\boldsymbol{h}\ |\ \boldsymbol{v})$ for some arbitrary $q$, then we get the $\log p(\boldsymbol{v}\ |\ \boldsymbol{h})$ and $\log p(\boldsymbol{h})$ term after applying Bayes rule to $\log p(\boldsymbol{h}\ |\ \boldsymbol{v})$. Here we can take the $\log p(\boldsymbol{v})$ term out of the expectation since it has no dependency with $\boldsymbol{h}$.\\
\begin{equation}
  \begin{split}
  	& D_{KL}\bigg[q(\boldsymbol{h}\ |\ \boldsymbol{v})\ ||\ p(\boldsymbol{h}\ |\ \boldsymbol{v})\bigg] \\
  	& = \mathbb{E}_{\boldsymbol{h}\sim q}\bigg[\log \frac{q(\boldsymbol{h}\ |\ \boldsymbol{v})}{p(\boldsymbol{h}\ |\ \boldsymbol{v})}\bigg]\\
  	& = \mathbb{E}_{\boldsymbol{h}\sim q}\bigg[\log q(\boldsymbol{h}\ |\ \boldsymbol{v}) - \frac{\log p(\boldsymbol{h}, \boldsymbol{v})}{}\bigg]\\
    & = \mathbb{E}_{\boldsymbol{h}\sim q}\bigg[\log q(\boldsymbol{h}) - \log \frac{p(\boldsymbol{v},\boldsymbol{h})}{p(\boldsymbol{v})}\bigg]\\
    & = \mathbb{E}_{\boldsymbol{h}\sim q}\bigg[\log q(\boldsymbol{h}) - \log \frac{p(\boldsymbol{v},\boldsymbol{h})}{p(\boldsymbol{v})}\bigg]\\
    & = \mathbb{E}_{\boldsymbol{h}\sim q}\bigg[\log q(\boldsymbol{h}) - \log \frac{p(\boldsymbol{v}\ |\ \boldsymbol{h})p(\boldsymbol{h})}{p(\boldsymbol{v})}\bigg]\\
    & = \mathbb{E}_{\boldsymbol{h}\sim q}\bigg[\log q(\boldsymbol{h}) - \log p(\boldsymbol{v}\ |\ \boldsymbol{h}) - \log p(\boldsymbol{h})\bigg] + \log p(\boldsymbol{v})\\
  \end{split}  
\end{equation}

After reforming the expectation term into KL divergence and reranging the formula, it yields the new objective function,
\begin{equation}
  \begin{split}
	J &= \log p(\boldsymbol{v}) - D_{KL}\bigg[q(\boldsymbol{h})\ ||\ p(\boldsymbol{h}\ |\ \boldsymbol{v})\bigg]\\
  &=\mathbb{E}_{\boldsymbol{h}\sim q}\bigg[\log p(\boldsymbol{v}\ |\ \boldsymbol{h})\bigg]-D_{KL}\bigg[q(\boldsymbol{h})\ ||\ p(\boldsymbol{h})\bigg].
  \end{split}  
\end{equation}

The left part of the equation has the log-likelihood $p(\boldsymbol{v})$ term we want to maximize plus an error term, which measures the difference of the distribution $q(\boldsymbol{h}\ |\ \boldsymbol{v})$ and the posterior $p(\boldsymbol{h}\ |\ \boldsymbol{v})$. Hence, the difference between our new objective function and the log-likelihood is decided by the KL divergence and these two are equal if and only if distribution $q$ is exactly the same as $p(\boldsymbol{h}\ |\ \boldsymbol{v})$. Because the KL divergence is always non-negative and this new objective function is therefore smaller than or at most equal to the real log-likelihood. In this case, the new objective function is defined as a lower bound $\mathcal{L}(\boldsymbol{v}, \theta, q)$ of log-likelihood function and this lower bound is called the $evidence\ lower\ bound$ (ELBO).\\

It is not hard to see, our new objective function is much easier to compute for some appropriate choice of distribution $q$. For any distribution $q$, our objective function is guaranteed to be the lower bound of log-likelihood and with a better approximation of posterior the lower bound will be closer to the real log-likelihood.
\subsection{Core Idea}
As shown in previous subsection, we can transfer the inference to a new optimization problem, in which we should find a proper distribution $q$ that maximize our objective function derived in the previous subsection. \\

In other words, the original intractable inference problem, which explicitly maximize the log-likelihood, can be thought as a new procedure that is less computational expensive  by using a approximated version of posterior from a restricted search family, which is imperfect approximate and may not completely maximize $\mathcal{L}$ but can improve it with a significantly amount.\\

To summarize, the core idea behind variational inference is that we can maximize the objective function over a restricted space of distribution $q$. This space should be chosen so that it makes the computation 
\subsection{Structure of Distribution $q$}
Another critical task in variational inference is to design a reparameteric distribution $q$ such that it can make a good approximation of posterior $q$ while still remains $\mathbb{E}_{\boldsymbol{h}\sim q}\bigg[p(\boldsymbol{v}\ |\ \boldsymbol{h})\bigg]$ feasible to compute. It could be very difficult in terms of reparameterization tricks, because an unproper reparameterization would make the distribution $q$ either a bad approximation to posterior or let $\mathbb{E}_{\boldsymbol{h}\sim q}\bigg[p(\boldsymbol{v}\ |\ \boldsymbol{h})\bigg]$ be to complex to solve.\\

In this subsection we provide two efficient approches to form a reasonable distribution $q$:
\subsubsection{Mean Field Approach}
The idea of Mean Field Approach is to compute the distribution $q$ given $\boldsymbol{v}$ by multiplying all the conditional probalities of single latent variable together.
\begin{equation}
	q(\boldsymbol{h}\ |\ \boldsymbol{v}) = \prod q(h_i\ |\ \boldsymbol{v})
\end{equation}

This approach is based on the assumption that all the relationships between the latent varialbes have been removed. Hence, we can impose the restriction that distribution $q$ is a factorial distribution. With this technique, the engineer just need to design how does distribution $q$ factorize rather than guess a accurate approximation distribution $q$, which is very likely to posterior. It helps to greatly reduce the work of algorithm designer.
\subsubsection{Deep Neural Network Approach}
A deep neural network is built in this case as a part of the distribution approximator, which takes the visialbe varaible $\boldsymbol{v}$ as the input and output the latent variable $\boldsymbol{h}$ with respect to some distribution $q$ (e.g. Gaussian Distribution).\\

The advantage is that this approach is a purely model-free approach and the desiner even doesn't need to concern about any structure issues. Therefore, it is also the most state-of-the-art approach in variational inference. With this end-to-end solution, the distribution approximator (or encoder) is able to learn the model parameters using backpropagation algorithm from the gradient information derived from the overall cost function. At last, we give a variational auto-encoder to show it has good performance not only in learning from the dataset, but also can generate new dataset with learned model.
\subsection{Variational Auto-Encoder}
Here we give an example of a variational auto-encoder presented in \cite{kingma2013auto}.
\begin{figure}[ht]
  \centering
  \includegraphics[scale=0.4]{picture3.png}
     \caption{RBM: Every latent variable is independent to each other since there is no connection between them. The posteriors are through factorization solvable.}\label{pic_3}
\end{figure}
\section{Experiment}
\begin{figure}[ht]
  \centering
  \includegraphics[scale=0.25]{picture4.png}
     \caption{RBM: Every latent variable is independent to each other since there is no connection between them. The posteriors are through factorization solvable.}\label{pic_4}
\end{figure}
% needed in second column of first page if using \IEEEpubid{}
%\IEEEpubidadjcol

% An example of a floating figure using the graphicx package.
% Note that \label must occur AFTER (or within) \caption.
% For figures, \caption should occur after the \includegraphics.
% Note that IEEEtran v1.7 and later has special internal code that
% is designed to preserve the operation of \label within \caption
% even when the captionsoff option is in effect. However, because
% of issues like this, it may be the safest practice to put all your
% \label just after \caption rather than within \caption{}.
%
% Reminder: the "draftcls" or "draftclsnofoot", not "draft", class
% option should be used if it is desired that the figures are to be
% displayed while in draft mode.
%
%\begin{figure}[!t]
%\centering
%\includegraphics[width=2.5in]{myfigure}
% where an .eps filename suffix will be assumed under latex, 
% and a .pdf suffix will be assumed for pdflatex; or what has been declared
% via \DeclareGraphicsExtensions.
%\caption{Simulation Results}
%\label{fig_sim}
%\end{figure}

% Note that IEEE typically puts floats only at the top, even when this
% results in a large percentage of a column being occupied by floats.


% An example of a double column floating figure using two subfigures.
% (The subfig.sty package must be loaded for this to work.)
% The subfigure \label commands are set within each subfloat command, the
% \label for the overall figure must come after \caption.
% \hfil must be used as a separator to get equal spacing.
% The subfigure.sty package works much the same way, except \subfigure is
% used instead of \subfloat.
%
%\begin{figure*}[!t]
%\centerline{\subfloat[Case I]\includegraphics[width=2.5in]{subfigcase1}%
%\label{fig_first_case}}
%\hfil
%\subfloat[Case II]{\includegraphics[width=2.5in]{subfigcase2}%
%\label{fig_second_case}}}
%\caption{Simulation results}
%\label{fig_sim}
%\end{figure*}
%
% Note that often IEEE papers with subfigures do not employ subfigure
% captions (using the optional argument to \subfloat), but instead will
% reference/describe all of them (a), (b), etc., within the main caption.


% An example of a floating table. Note that, for IEEE style tables, the 
% \caption command should come BEFORE the table. Table text will default to
% \footnotesize as IEEE normally uses this smaller font for tables.
% The \label must come after \caption as always.
%
%\begin{table}[!t]
%% increase table row spacing, adjust to taste
%\renewcommand{\arraystretch}{1.3}
% if using array.sty, it might be a good idea to tweak the value of
% \extrarowheight as needed to properly center the text within the cells
%\caption{An Example of a Table}
%\label{table_example}
%\centering
%% Some packages, such as MDW tools, offer better commands for making tables
%% than the plain LaTeX2e tabular which is used here.
%\begin{tabular}{|c||c|}
%\hline
%One & Two\\
%\hline
%Three & Four\\
%\hline
%\end{tabular}
%\end{table}


% Note that IEEE does not put floats in the very first column - or typically
% anywhere on the first page for that matter. Also, in-text middle ("here")
% positioning is not used. Most IEEE journals use top floats exclusively.
% Note that, LaTeX2e, unlike IEEE journals, places footnotes above bottom
% floats. This can be corrected via the \fnbelowfloat command of the
% stfloats package.

% if have a single appendix:
%\appendix[Proof of the Zonklar Equations]
% or
%\appendix  % for no appendix heading
% do not use \section anymore after \appendix, only \section*
% is possibly needed

% use appendices with more than one appendix
% then use \section to start each appendix
% you must declare a \section before using any
% \subsection or using \label (\appendices by itself
% starts a section numbered zero.)
%

  
%\appendices
%\section{Proof of the First Zonklar Equation}

% use section* for acknowledgement
%\section*{Acknowledgment}


%The authors would like to thank..


% Can use something like this to put references on a page
% by themselves when using endfloat and the captionsoff option.
% \ifCLASSOPTIONcaptionsoff
%   \newpage
% \fi


\newpage
% trigger a \newpage just before the given reference
% number - used to balance the columns on the last page
% adjust value as needed - may need to be readjusted if
% the document is modified later
%\IEEEtriggeratref{8}
% The "triggered" command can be changed if desired:
%\IEEEtriggercmd{\enlargethispage{-5in}}

% references section

% can use a bibliography generated by BibTeX as a .bbl file
% BibTeX documentation can be easily obtained at:
% http://www.ctan.org/tex-archive/biblio/bibtex/contrib/doc/
% The IEEEtran BibTeX style support page is at:
% http://www.michaelshell.org/tex/ieeetran/bibtex/
%\bibliographystyle{IEEEtran}
% argument is your BibTeX string definitions and bibliography database(s)
%\bibliography{IEEEabrv,../bib/paper}
%
% <OR> manually copy in the resultant .bbl file
% set second argument of \begin to the number of references
% (used to reserve space for the reference number labels box)
\bibliographystyle{IEEEtran}
\bibliography{IEEEabrv,reference}

% biography section
% 
% If you have an EPS/PDF photo (graphicx package needed) extra braces are
% needed around the contents of the optional argument to biography to prevent
% the LaTeX parser from getting confused when it sees the complicated
% \includegraphics command within an optional argument. (You could create
% your own custom macro containing the \includegraphics command to make things
% simpler here.)
%\begin{biography}[{\includegraphics[width=1in,height=1.25in,clip,keepaspectratio]{mshell}}]{Michael Shell}
% or if you just want to reserve a space for a photo:

% \begin{IEEEbiography}[{\includegraphics[width=1in,height=1.25in,clip,keepaspectratio]{picture}}]{John Doe}
% % \blindtext
% [1] R. S. Sutton and A. G. Barto, Reinforcement learning: An introduction, 2nd ed. Cambridge, MA: MIT Press, 1998.
% \end{IEEEbiography}

% You can push biographies down or up by placing
% a \vfill before or after them. The appropriate
% use of \vfill depends on what kind of text is
% on the last page and whether or not the columns
% are being equalized.

%\vfill

% Can be used to pull up biographies so that the bottom of the last one
% is flush with the other column.
%\enlargethispage{-5in}




% that's all folks
\end{document}


